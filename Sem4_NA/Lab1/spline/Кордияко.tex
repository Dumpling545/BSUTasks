\documentclass[fleqn]{article}
\usepackage{cmap}
\usepackage[T2A]{fontenc}
\usepackage[utf8]{inputenc}
\usepackage[russian]{babel}
\usepackage[left=2cm,right=2cm,top=2cm,bottom=3cm,bindingoffset=0cm]{geometry}
\usepackage{subfiles}
\usepackage[bookmarks]{hyperref}


\usepackage[fleqn]{amsmath}
\usepackage{amssymb}
\usepackage{amsthm}
\usepackage{color}
\usepackage{cmap}
\usepackage{graphicx}
\usepackage{indentfirst}
\usepackage{titling}
\usepackage{verbatim}
\usepackage{enumitem}
\usepackage{esint}
\usepackage{etoolbox}
\usepackage{soul}
\usepackage{cancel}
\usepackage{centernot}
\DeclareMathOperator\Ln{Ln}
\DeclareMathOperator\Imaginary{Im}
\DeclareMathOperator\Real{Re}
\begin{document}

Кордияко Ян, 2 курс, 1 группа

Вариант 12

\textbf {Задание 1}. Доказать, что оператор $A: C[1, 5] \xrightarrow{} C[1, 5]$, является линейным
ограниченным и найти его норму:
\begin{equation*} 
	Ax(t) = (t^3 - 3t)x(t).
\end{equation*}

Очевидно, что оператор $A$ линеен:
\begin{equation*} 
	A(\alpha{}x(t) + \beta{}y(t)) = (t^3 - 3t)(\alpha{}x(t) + \beta{}y(t)) = 
	\alpha{}(t^3 - 3t)x(t) + \beta{}(t^3 - 3t)y(t) = \alpha{}Ax(t) + \beta{}Ay(t)
\end{equation*}

Покажем, что оператор $A$ ограничен:
\begin{equation*} 
	||Ax(t)||_{ C[1, 5]} = \max_{1 \leq t \leq 5}{|Ax(t)|}  = \max_{1 \leq t \leq 5}{|(t^3 - 3t)x(t)|} \leq
	\max_{1 \leq t \leq 5}{|(t^3 - 3t)|}\max_{1 \leq t \leq 5}{|x(t)|} = 110||x(t)||_{ C[1, 5]}
\end{equation*}

Так как взяв функцию $x_0(t) = 1$, мы получим выше равенство,  норма оператора $||A|| = 110$, более того, она достижима.

\textbf {Задание 2}. Доказать, что оператор $A: L_5[-1, 1] \xrightarrow{} L_5[-1, 1]$, является линейным
ограниченным и найти его норму:
\begin{equation*} 
	Ax(t) = (t^2 - t)x(t^5).
\end{equation*}

Очевидно, что оператор $A$ линеен:
\begin{equation*} 
	A(\alpha{}x(t) + \beta{}y(t)) = (t^2 - t)(\alpha{}x(t^5) + \beta{}y(t^5)) = 
	\alpha{} (t^2 - t)x(t^5) + \beta{} (t^2 - t)y(t^5) = \alpha{}Ax(t) + \beta{}Ay(t)
\end{equation*}

Покажем, что оператор $A$ ограничен:
\begin{equation*} 
	||Ax(t)||_{ L_5[-1, 1]} = \left(\int_{-1}^1 |Ax(t)|^5 dt\right)^{1/5} = 
	\left(\int_{-1}^1 |t - 1|^5 |t|^5 |x(t^5)|^5 dt\right)^{1/5} =
\end{equation*}
\begin{equation*} 
	= 5^{-1/5}\left(\int_{-1}^1 |s^{1/5} - 1|^5 |s|^{1/5} |x(s)|^5 ds\right)^{1/5}
	 \leq
	 5^{-1/5}\left(\int_{-1}^1 \max_{-1 \leq r \leq 1}{(|r^{1/5} - 1|^5 |r|^{1/5})} |x(s)|^5 ds\right)^{1/5} = 
\end{equation*}
\begin{equation*} 
	= 
	\frac{5^{4/5}}{6^{6/5}}\cdot{}\left(\int_{-1}^1 |x(s)|^5 ds\right)^{1/5} = 
	\frac{5^{4/5}}{6^{6/5}}\cdot{}||x(t)||_{ L_5[-1, 1]}
\end{equation*}
	
	Следовательно, $||A|| \leq \frac{5^{4/5}}{6^{6/5}}$. 
	С другой стороны, $||A|| \geq \frac{||Ax||}{||x||}$ для всех $x(t) \in  L_5[-1, 1]$. 
	Выберем подпоследовательность
	\begin{equation*} 
		x_n(t) = 	\begin{cases}
					0, t \in [-1, \frac1{7776} - \frac1n), \\
					1, t \in [\frac1{7776} - \frac1n, \frac1{7776}), \\
					0, t \in [\frac1{7776}, 1]
				\end{cases}
	\end{equation*}
	
	норма которой:
	\begin{equation*} 
		||x_n(t)||_{ L_5[-1, 1]} = \left(\int_{\frac1{7776} - \frac1n}^\frac1{7776} dt\right)^{1/5} = n^{-1/5}
	\end{equation*}
	Имеем:
	\begin{equation*} 
	||Ax_n(t)||_{ L_5[-1, 1]} = \left(\int_{\frac1{7776} - \frac1n}^\frac1{7776}  |t - 1|^5 |t|^5 dt\right)^{1/5} = 
	 \left(\int_{\frac1{7776} - \frac1n}^\frac1{7776}  (1 - t)^5 t^5 dt\right)^{1/5}
	\end{equation*}
\textbf {Задание 3}. Доказать, что оператор $A: C[-1, 2] \xrightarrow{} C[0, 3]$, является линейным
ограниченным и найти его норму:
\begin{equation*} 
	Ax(t) = \int_{-1}^1 s^3(t^3 - t)x(s)ds.
\end{equation*}
	Очевидно, что оператор $A$ линеен:
\begin{equation*} 
	A(\alpha{}x(t) + \beta{}y(t)) = \int_{-1}^1 s^3(t^3 - t)(\alpha{}x(s) + \beta{}y(s))ds = 
	\alpha\int_{-1}^1 s^3(t^3 - t)x(s)ds + \beta\int_{-1}^1 s^3(t^3 - t)y(s)ds = \alpha{}Ax(t) + \beta{}Ay(t)
\end{equation*}
Покажем, что оператор $A$ ограничен:

\begin{equation*} 
	||Ax||_ {C[0, 3]} = \max_{0 \leq t \leq 3}{|Ax(t)|} =  
	\max_{0 \leq t \leq 3}{\left|\int_{-1}^1 s^3(t^3 - t)x(s)ds\right|} \leq
	\max_{0 \leq t \leq 3}{|t^3 - t|}\cdot{}\int_{-1}^1 |s^3||x(s)|ds \leq
	24\cdot{}\int_{-1}^1 |s^3|\max_{-1 \leq u \leq 1}{|x(u)|}ds = 
\end{equation*}	
\begin{equation*} 
	= 24\cdot{}\max_{-1 \leq u \leq 1}{|x(u)|}\cdot{}2\cdot\int_{0}^1 s^3 ds \leq
	12\cdot{}\max_{-1 \leq u \leq 2}{|x(u)|} = 12||x||_{C[-1, 2]}
\end{equation*}

	Следовательно, $||A|| \leq 12$. Покажем, что $||A|| \geq 12$. Заметим, что
	при любом фиксированном $t \in [-1, 2]$ ядро $K(t, s) = (t^3 - t)*s^3$ интегрального оператора по 	переменнной $s \in [-1, 1]$ меняет знак, поэтому построим последовательность $x_n(t)$ вида:
	\begin{equation*} 
	x_n(t) = 	\begin{cases}
					-1, t \in [-1,-\frac1n), \\
					nt, t \in [-\frac1n, \frac1n], \\
					1, t \in (\frac1n, 2]
				\end{cases}
\end{equation*}
	с нормой $||x_n||_{C[-1, 2]} = 1$. Тогда:
	\begin{equation*} 
	||A|| \geq ||Ax_n||_{C[0, 3]} =  \max_{0 \leq t \leq 3}{|Ax_n(t)|} =
	\max_{0 \leq t \leq 3}{|t^3 - t|}\left|\int_{-1}^{-\frac1n} -s^3 ds + 
	\int_{-\frac1n}^{\frac1n} s^3ns ds + \int_{\frac1n}^{1} s^3 ds\right| =
	\end{equation*}
	\begin{equation*} 
	 =\max_{0 \leq t \leq 3}{\left|(t^3 - t)\left(\frac12 - \frac1{2n^4} + \frac2{5n^4}\right)\right|} =
	 12 - O(\frac1{n^4}).
	\end{equation*}
	Следовательно, $||A|| = 12$.
	
\textbf {Задание 4}. Вычислить норму оператора $A: L_2[-1, 1] \xrightarrow{} L_1[0, 3]$:
\begin{equation*} 
	Ax(t) = \int_{-1}^1 s^3(1 - t)x(s)ds.
\end{equation*}

\begin{equation*} 
	||Ax||_{ L_1[0, 3]} = \int_0^3 \left| \int_{-1}^1 s^3(1 - t)x(s)ds\right| dt =
	 \left| \int_{-1}^1 s^3x(s)ds\right|  \int_0^3 |1 - t| dt = \frac52 \cdot  \left| \int_{-1}^1 s^3x(s)ds\right| \leq
	 \frac52 \cdot \int_{-1}^1 |s^3x(s)|ds \leq
\end{equation*}
\begin{equation*} 
	\leq \left[\text{Неравенство Гёльдера, } p = q = 2\right]
	 \leq  \frac52 \cdot \left(\int_{-1}^1 s^6 ds \right)^{1/2} \cdot
	  \left(\int_{-1}^1 |x(s)|^2 ds \right)^{1/2} =
	  \frac5{\sqrt{14}}\cdot{}||x||_{L_2[-1, 1]}
\end{equation*}

То есть $||A|| \leq  \frac5{\sqrt{14}}$.Покажем, 
что $||A|| \geq  \frac5{\sqrt{14}}$. 
Выберем в качестве функции $x(t)$ функцию $x_0(t) = t^3, t\in [-1, 1]$, поскольку именно
для такой функции неравенство Гёльдера, которое было использовано выше при проведении оценок,
обратится в равенство. Тогда:
\begin{equation*} 
	||Ax_0||_{ L_1[0, 3]} = \int_0^3 \left| \int_{-1}^1 s^3(1 - t)s^3ds\right| dt =
	   \frac52 \cdot  \left| \int_{-1}^1 s^6 ds\right| = \frac57
\end{equation*}
\begin{equation*} 
	||x_0||_{L_2[-1, 1]} =  \left(\int_{-1}^1 s^6 ds \right)^{1/2} = \sqrt{\frac27}
\end{equation*}
\begin{equation*} 
	||A|| \geq \frac{||Ax_0||_{ L_1[0, 3]}}{||x_0||_{L_2[-1, 1]}} = \frac{\frac57}{\sqrt{\frac27}} =  \frac5{\sqrt{14}}
\end{equation*}
Следовательно, $||A|| = \frac5{\sqrt{14}}$.

\textbf {Задание 5}. Вычислить норму оператора $A: C[-1, 1] \xrightarrow{} L_{3/2}[0, 1]$:
\begin{equation*} 
	Ax(t) = \int_{-1}^1 s^3(1 - t)x(s)ds + tx(0).
\end{equation*}

\begin{equation*} 
	||Ax||_{ L_2[0, 1]} =\left(\int_0^1 |Ax(t)|^{3/2} dt\right)^{2/3} = 
	\left(\int_0^1 \left|(1 - t)\int_{-1}^1 s^3x(s)ds + tx(0)\right|^{3/2} dt\right)^{2/3}
\end{equation*}

Оценим выражение, стоящее под знаком модуля:

\begin{equation*} 
	\left|(1 - t)\int_{-1}^1 s^3x(s)ds + tx(0)\right| \leq 
	|1 - t|\int_{-1}^1 |s^3||x(s)|ds + |t||x(0)|\leq 
	|1 - t|\int_{-1}^1 |s^3|\max_{-1 \leq u \leq 1}{|x(u)|}ds + |t|\max_{-1 \leq u \leq 1}{|x(u)|}\leq 
\end{equation*}
\begin{equation*} 
	\leq \max_{-1 \leq u \leq 1}{|x(u)|}\left(\frac{|1 - t|}2 + |t|\right)=
	\left[\quad t\in [0,1] \quad\right] =\frac{1 + t}2\cdot{||x||_{C[-1,1]}},\quad \text{для }\quad t \in [0, 1]
\end{equation*}

Полученную оценку подставим в выражение для нормы $||Ax||_{ L_2[0, 1]}$:
\begin{equation*} 
	||Ax||_{ L_2[0, 1]} \leq \left(\int_0^1 \left(\frac{1 + t}2\right)^{3/2} dt\right)^{2/3}\cdot{||x||_{C[-1,1]}}
\end{equation*}

Откуда следует, что $||A|| \leq \left(\int_0^1 \left(\frac{1 + t}2\right)^{3/2} dt\right)^{2/3}$. 
Для доказательства неравеснства в обратную сторону потроим
последовательность
\begin{equation*} 
	x_n(t) = 	\begin{cases}
			-1, t \in [-1,-\frac1n], \\
			-2nt + 1, t \in (-\frac1n, 0), \\
			1, t \in [0, 1]
		\end{cases}
\end{equation*}
с нормой $||x_n||_{C[-1,1]} = 1$. Тогда:
\begin{equation*} 
	||A|| \geq ||Ax_n(t)|| = 
	\left(\int_0^1 \left|(1 - t)\left(\int_{-1}^{-1/n} -s^3ds + \int_{-1/n}^{0} s^3(-2ns + 1)ds + \int_{0}^1 s^3ds\right) + t\right|^{3/2} dt\right)^{2/3} =
\end{equation*}
\begin{equation*} 
	 = \left(\int_0^1 \left|(1 - t)\left(\frac14 - \frac1{4n^4} - \frac{13}{20 n^4} + \frac14\right) + t\right|^{3/2} dt\right)^{2/3} = \left(\int_0^1 \left(\frac{1 + t}2\right)^{3/2} dt\right)^{2/3} - O\left(\frac1n\right)
\end{equation*}
	Это означает, что $||A|| = \left(\int_0^1 \left(\frac{1 + t}2\right)^{3/2} dt\right)^{2/3} = \left(\frac1{10}\cdot{}(8 - \sqrt{2})\right)^{2/3}$
	
	\textbf {Задание 6}. Вычислить норму оператора $A: l_2\xrightarrow{}l_2$:
\begin{equation*} 
	Ax = \left(\frac{x_1\sin1}{3},\frac{x_2\sin2}{3^2}, ..., \frac{x_k\sin{k}}{3^k}, ... \right).
\end{equation*}

	Так как последовательность $\frac{\sin{k}}{3^k}$ ограничена, то  $\exists\sup_k{\left|\frac{\sin{k}}{3^k}\right|} = \frac{sin 1}{3}$. Тогда:
	\begin{equation*} 
	||Ax||_{l_2} = \left(\sum_{k=1}^{\infty}\left(\frac{x_k\sin{k}}{3^k}\right)^2\right)^{1/2} =
	\left(\sum_{k=1}^{\infty}\left(\frac{\sin{k}}{3^k}\right)^2x_k^2\right)^{1/2} \leq
	\left(\sum_{k=1}^{\infty}\left(\sup_n{\left|\frac{\sin{n}}{3^n}\right|}\right)^2x_k^2\right)^{1/2} = 
\end{equation*}
\begin{equation*} 
	= \frac{sin 1}{3} \cdot{\left(\sum_{k=1}^{\infty}x_k^2\right)^{1/2} } = \frac{sin 1}{3} \cdot{||x||_{l_2}}.
\end{equation*}
	Возьмём последовательность $x_0 = (1, 0, 0, ..., 0, ...)$. Очевидно, что для неё 
	$||x_0||_{l_2} = 1$. Посчитаем $||Ax_0||_{l_2}$:
	\begin{equation*} 
	||Ax_0||_{l_2} = \frac{\sin{1}}{3}
\end{equation*}

Тогда $||A|| = \frac{\sin{1}}{3}$. Более того, норма достижима.
\end{document}
