\documentclass[fleqn]{article}
\usepackage{cmap}
\usepackage[T2A]{fontenc}
\usepackage[utf8]{inputenc}
\usepackage[russian]{babel}
\usepackage[left=2cm,right=2cm,top=2cm,bottom=3cm,bindingoffset=0cm]{geometry}
\usepackage{subfiles}
\usepackage[bookmarks]{hyperref}


\usepackage[fleqn]{amsmath}
\usepackage{amssymb}
\usepackage{amsthm}
\usepackage{color}
\usepackage{cmap}
\usepackage{graphicx}
\usepackage{indentfirst}
\usepackage{titling}
\usepackage{verbatim}
\usepackage{enumitem}
\usepackage{esint}
\usepackage{etoolbox}
\usepackage{soul}
\usepackage{cancel}
\usepackage{centernot}
\DeclareMathOperator\Ln{Ln}
\DeclareMathOperator\Imaginary{Im}
\DeclareMathOperator\Real{Re}
\begin{document}

Кордияко Ян, 2 курс, 1 группа

\textbf {Задание 3}. Построить квадратуру Гаусса с тремя узлами для вычисления интеграла
$I(f) = \int_{-1}^1 f(x) dx$.

Квадратура Гаусса по $n = 3$\  \  узлам имеет вид:
\begin{equation*}
	S_3(f) = D_1\cdot{}f(x_1) + D_2\cdot{}f(x_2) + D_3\cdot{}f(x_3)
\end{equation*}
По определению, квадратура точна для всех многочленов степени $2n - 1 = 5$. Иначе
говоря, для определения $D_1, D_2, D_3,\  x_1, x_2, x_3$ необходимо решить систему:
\begin{equation*}
	\begin{cases}
	D_1 + D_2 + D_3 =   \int_{-1}^1 1 dx= 2\\
	D_1\cdot{}x_1 + D_2\cdot{}x_2 + D_3\cdot{}x_3 =   \int_{-1}^1 x dx = 0\\
	D_1\cdot{}x_1^2 + D_2\cdot{}x_2^2 + D_3\cdot{}x_3^2 =   \int_{-1}^1 x^2 dx = \frac23\\
	D_1\cdot{}x_1^3 + D_2\cdot{}x_2^3 + D_3\cdot{}x_3^3 =   \int_{-1}^1 x^3  dx = 0\\
	D_1\cdot{}x_1^4 + D_2\cdot{}x_2^4 + D_3\cdot{}x_3^4 =   \int_{-1}^1 x^4 dx = \frac25\\
	D_1\cdot{}x_1^5 + D_2\cdot{}x_2^5 + D_3\cdot{}x_3^5 =   \int_{-1}^1 x^5 dx = 0\\
	\end{cases}
\end{equation*}

В силу того, что функцией веса является $p(x) \equiv 1$,  искомые $D_1, D_2, D_3$, а
также числа $d_1, d_2, d_3$, вычисляемые по формуле $d_j = \frac{2x_j - (a+b)}{b - a}$,
не зависят от отрезка интегрирования $[a, b]$. Более того, для отрезка
$[-1, 1]$ получаем, что $d_j = x_j$. Поэтому мы можем воспользоваться 
предвычисленными значениями из \textbf {Таблицы 1} 
(\emph{Бахвалов, Жидков, Кобельков. Численные методы}, стр. 110):
\begin{equation*}
	\begin{cases}
	D_1  = D_3 \approx 0.5555555556 \\
	D_2 \approx 0.8888888888 \\
	x_1 \approx - 0.7745966692 \\
	x_2 = 0 \\
	x_3 \approx  0.7745966692
	\end{cases}
\end{equation*}
Таким образом, искомая квадратура Гаусса примет вид: 
\begin{equation*}
	S_3(f) = 0.5555555556\cdot{}f(- 0.7745966692) +
	0.8888888888\cdot{}f(0) + 
	0.5555555556\cdot{}f(0.7745966692).
\end{equation*}
\end{document}
