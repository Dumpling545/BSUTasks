\documentclass[fleqn]{article}
\usepackage{cmap}
\usepackage[T2A]{fontenc}
\usepackage[utf8]{inputenc}
\usepackage[russian]{babel}
\usepackage[left=2cm,right=2cm,top=2cm,bottom=3cm,bindingoffset=0cm]{geometry}
\usepackage{subfiles}
\usepackage[bookmarks]{hyperref}


\usepackage[fleqn]{amsmath}
\usepackage{amssymb}
\usepackage{amsthm}
\usepackage{color}
\usepackage{cmap}
\usepackage{graphicx}
\usepackage{indentfirst}
\usepackage{titling}
\usepackage{verbatim}
\usepackage{enumitem}
\usepackage{esint}
\usepackage{etoolbox}
\usepackage{soul}
\usepackage{cancel}
\usepackage{centernot}
\DeclareMathOperator\Ln{Ln}
\DeclareMathOperator\Imaginary{Im}
\DeclareMathOperator\Real{Re}
\begin{document}

Кордияко Ян, 2 курс, 1 группа

Вариант 12

\textbf {Задание 1}. Найти сопряжённый оператор $A^*$ к оператору $A : L_2[0,1] \rightarrow L_2[0,1]$,
действующему по следующей формуле:
\begin{equation*} 
	Ax(t) = \int_{t^3}^1 t^2 x\left(s^{\frac13}\right)ds + \int_0^{t^2} t s x(s) ds
\end{equation*}
	По определению сопряжённого оператора имеем
\begin{equation*} 
	f(Ax) = (Ax, y)_{L_2[0,1]} = \int_0^1 Ax(t)y(t) dt =  \int_0^1 \left( \int_{t^3}^1 t^2 x\left(s^{\frac13}\right)ds + \int_0^{t^2} t s x(s) ds\right)y(t) dt = \int_0^1 \int_{t^3}^1 t^2 x\left(s^{\frac13}\right) y(t)  ds dt\   +
\end{equation*}	
\begin{equation*} 
	 +
	\int_0^1 \int_0^{t^2} t s x(s) y(t) ds dt =  \int_0^1 x\left(s^{\frac13}\right) \left(\int_{0}^{s^{\frac13}} t^2  y(t)  dt\right) ds + \int_0^1 x(s)\left(\int_{s^{\frac12}}^1 t s  y(t) dt\right) ds = [\text{замена в первом интеграле}] =
\end{equation*}	
\begin{equation*} 
	=
	\int_0^1 x(s) \left(\int_0^s 3s^2 t^2  y(t)  dt\right) ds + \int_0^1 x(s)\left(\int_{s^{\frac12}}^1 t s  y(t) dt\right) ds =
	\int_0^1 x(s) \left(\int_0^s 3s^2 t^2  y(t)  dt  + \int_{s^{\frac12}}^1 t s  y(t) dt\right) ds = (x, A^*y).
\end{equation*}	
Откуда:
\begin{equation*} 
	A^*y(t) = \int_0^t 3t^2 s^2  y(s)  ds  + \int_{t^{\frac12}}^1 s t  y(s) ds.
\end{equation*}
\textbf {Задание 2}. Найти сопряжённый оператор $A^*$ к оператору $A : l_2 \rightarrow l_2$,
действующему по следующей формуле:
\begin{equation*} 
	Ax = (x_2 + x_1, x_1 - x_2, x_4, x_3, x_5, x_6, ...), \quad x = (x_1, x_2, ...) \in l_2
\end{equation*}
По определению сопряжённого оператора имеем
\begin{equation*} 
	f(Ax) = (Ax, y)_{l_2} = \sum_{i = 1}^{\infty}{Ax_i\cdot{y_i}} = (x_2 + x_1)\cdot{y_1} + 
	(x_1 - x_2)\cdot{y_2} + x_4\cdot{y_3} + x_3\cdot{y_4} + \sum_{i = 5}^{\infty}{x_i\cdot{y_i}} = 
\end{equation*}
\begin{equation*} 
	= x_1\cdot{}(y_1 + y_2) +  x_2\cdot{}(y_1 - y_2) + x_3\cdot{}y_4 + x_4\cdot{}y_3 + \sum_{i = 5}^{\infty}{x_i\cdot{y_i}} = (x,A^*y)_{l_2}
\end{equation*}
Откуда:
\begin{equation*} 
	A^*y = (y_1 + y_2, y_1 - y_2, y_4, y_3, y_5, y_6, ...)
\end{equation*}
Таким образом оператор $A$ является самосопряжённым.
\end{document}
