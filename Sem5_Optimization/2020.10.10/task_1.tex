\documentclass[11pt]{article} 
\usepackage{amsmath}
\usepackage{amsthm}
\usepackage{amssymb}
\usepackage{mathtext}
\usepackage[T1,T2A]{fontenc}
\usepackage[utf8]{inputenc}
\usepackage[english,bulgarian,ukrainian,russian]{babel}
\usepackage{cases}
\usepackage{systeme}
\newenvironment{amatrix}[1]{%
  \left(\begin{array}{@{}*{#1}{c}|c@{}}
}{%
  \end{array}\right)
}
\begin{document}
\textbf{Задание 1}. Решить графически следующую задачу ЛП как на максимум, так и на минимум:

Целевая функция:\quad  \textbf{11.}\  $\phi = 4x_1 + 3x_2$

Ограничения:\quad \textbf{12.}\  $
\begin{array}{l}
\systeme{  x_2 \leq 4, -4x_1 + 7x_2 \geq - 28, 3x_1 + x_2 \geq 3}\\
x_1 \geq 0,\ x_2 \geq 0
\end{array}
$

 \hfill

Ответ: $min = (1; 0),\  max = (14; 4)$

Построим  \textbf{двойственную задачу}, используя
мнемоническое правило:
\begin{enumerate}
	\item Вводим новые переменные, соответствующие основным ограничениям задачи:

	$$y = (y_1, y_2, y_3)^T$$
	\item \begin{enumerate}
		\item $min \rightarrow{} max$
		\item $max \rightarrow{} min$
	\end{enumerate}
	\item $$A_{\text{двойственная}} = A^T = 
	\begin{bmatrix}
	0 & -4 & 3\\
	1 & 7 & 1
	\end{bmatrix}$$
	$$b_{\text{двойственная}} = c =\begin{bmatrix}
	4\\
	3
	\end{bmatrix}$$
	$$c_{\text{двойственная}} = b =\begin{bmatrix}
	4\\
	-28\\
	3
	\end{bmatrix}$$
	\item  \begin{enumerate}
		\item В случае  $min \rightarrow{} max$ 
		основные ограничения двойственной задачи будут иметь следующие знаки:
		\begin{enumerate}
			\item $\leq$ 
			\item $\leq$,
		\end{enumerate}
		а прямые ограничения:
		\begin{enumerate}
			\item $\leq$
			\item $\geq$
			\item $\geq$
		\end{enumerate}
		\item В случае   $max \rightarrow{} min$ 
		основные ограничения двойственной задачи будут иметь следующие знаки:
		\begin{enumerate}
			\item $\geq$ 
			\item $\geq$,
		\end{enumerate}
		а прямые ограничения:
		\begin{enumerate}
			\item $\geq$
			\item $\leq$
			\item $\leq$
		\end{enumerate}
	\end{enumerate}
\end{enumerate}
Таким образом, \textbf{задача, двойственная к задаче $min$}  имеет вид:
$$\phi_{\text{двойственная}} = 4y_1 -28y_2 + 3y_3 \rightarrow{} max$$

$$
\begin{array}{l}
\systeme{  -4y_2 + 3y_3 \leq 4, y_1 + 7y_2 + y_3 \leq 3}\\
y_1 \leq 0,\ y_2 \geq 0, \ y_3 \geq 0
\end{array}
$$

А \textbf{задача, двойственная к задаче $max$}  имеет вид:
$$\phi_{\text{двойственная}} = 4y_1 -28y_2 + 3y_3 \rightarrow{} min$$

$$
\begin{array}{l}
\systeme{  -4y_2 + 3y_3 \geq 4, y_1 + 7y_2 + y_3 \geq 3}\\
y_1 \geq 0,\ y_2 \leq 0, \ y_3 \leq 0
\end{array}
$$

Для того, чтобы получить ответ двойственной задачи,
приведём исходную задачу к каноническому виду:

$$
	\psi = -\phi(x) = - 4x_1 - 3x_2 \rightarrow \max{}
	\left(\equiv \phi(x) \rightarrow \min{}\right)
$$
(для задачи на максимум $\phi = 4x_1 + 3x_2 
\rightarrow \max{}$)
$$
\begin{array}{l}
\systeme{  x_2 + x_3 = 4, 
-4x_1 + 7x_2 - x_4 = - 28, 
 3x_1 + x_2 - x_5 = 3}\\
M \geq x_1 \geq 0,\ M \geq x_2 \geq 0,\\
M \geq x_3 \geq 0,\ M \geq x_4 \geq 0,\   
M \geq x_5 \geq 0
\end{array}
$$

Для получения ответа двойственной задачи к
задаче на минимум
построим вектор потенциалов $u$ оптимального
базисного плана. Для построения оптимального
найдём $x_3, x_4, x_5$:\  подставим в основные
ограничения полученный графическим методом
оптимальный план исходной задачи $(\ min = (1, 0)\ )$ :

$$\begin{amatrix}{3}
	1 &	0 & 0 & 4 \\
	0 & -1 & 0 & -24 \\
	0 & 0 & -1 & 0
\end{amatrix}$$

Таким образом, получаем план $x^* = (1, 0, 4, 24, 0)^T$. 
Данный план будет базисным с базисом $J_{\text{Б}}=\{1, 3, 4\}$, 
так 
как соответствующая матрица невырождена, а $x_3, x_5$ лежат
на нижней границе. Начинаем вторую фазу симплекс-метода:
$$\begin{amatrix}{3}
	0 &	-4 & 3 & -4 \\
	1 & 0 & 0 & 0 \\
	0 & -1 & 0 & 0
\end{amatrix} \implies
u_{min} = \begin{bmatrix}
	0\\0\\-\frac43
\end{bmatrix}
$$
$$
\Delta_2 = -4 + \frac43 = -\frac{8}3 < 0, \quad
\Delta_5 =  0 - \frac43 = -\frac43 < 0
$$
Следовательно базисный план оптимальный. В таком
случае $y_{min} = -u_{min}$ -- решение задачи, 
двойственной к задаче на минимум (меняем знак 
в силу того, что исходная задача была на
минимум). Проверим:
$$
	\phi(x^*) = -\psi(x^*) = 4\cdot{1} + 3\cdot{0} = 
	4 = 4\cdot{0} - 28\cdot{0} + 
	3\cdot{\frac43} =
	\phi_{двойственная}(y_{min})
$$

Проведём аналогичные операции с задачей на максимум 
(\ оптимальный план исходной задачи $max=(14, 4)$\ ):

$$\begin{amatrix}{3}
	1 &	0 & 0 & 0 \\
	0 & -1 & 0 & 0 \\
	0 & 0 & -1 & -43
\end{amatrix}$$

Таким образом, получаем план $x^{**} = (14, 4, 0, 0, 43)^T$. 
Данный план будет базисным с базисом $J_{\text{Б}}=\{1, 2, 5\}$, 
так 
как соответствующая матрица невырождена, а $x_3, x_4$ лежат
на нижней границе. Начинаем вторую фазу симплекс-метода:

$$\begin{amatrix}{3}
	0 &	-4 & 3 & 4 \\
	1 & 7 & 1 & 3 \\
	0 & 0 & -1 & 0
\end{amatrix} \implies 
u_{max} = \begin{bmatrix}
	10 \\ -1 \\ 0
\end{bmatrix}$$

$$
\Delta_3 =  0 - 10 = -10 < 0, \quad
\Delta_4 =  0 - 1 = -1 < 0
$$
Следовательно базисный план оптимальный. В таком
случае $y_{max} = u_{max}$ -- решение задачи, 
двойственной к задаче на максимум. Проверим:
$$
\phi(x^{**}) = 4\cdot{14} + 3\cdot{4} = 68 =
	4\cdot{10} - 28\cdot{(-1)} + 3\cdot{0} =
	\phi_{двойственная}(y_{max})
$$

\textbf{Ответ: } 
$\lambda^o_{min \rightarrow max} = (0, 0 , \frac43)^T,\  
\lambda^o_{max \rightarrow min} = (10, -1 , 0)^T$

\textbf{Задание 3}. Решить симплекс-методом следующую задачу ЛП:

Целевая функция:\quad   $\phi = 6x_1 + 2x_2 + 9x_3 + 2x_4 \rightarrow{} max$

Ограничения:\quad $
\begin{array}{l}
\systeme{  -x_1 +3x_3 = 13, 2x_2 + 4x_3 + x_5 = 49, 
5x_1 + 2x_4 = 42}\\
1 \leq x_1 \leq 8,\ 2 \leq x_2 \leq 10, 1 \leq x_3 \leq 8,\\
0 \leq x_4 \leq 6, 0 \leq x_5 \leq 7
\end{array}
$

 \hfill

Ответ:  $(8; 10; 7; 1; 1)$

Построим  \textbf{двойственную задачу к канонической задаче}:

Целевая функция:\quad   $$\psi(\lambda) = 13y_1 + 49y_2 + 42y_3 + 8w_1 + 10w_2 +
8w_3 + 6w_4 + 7w_5  - v_1 - 2v_2 - v_3\rightarrow{} min$$

Ограничения:  $
\begin{array}{l}
\syssubstitute{.,{a_1}{y_1}{a_2}{y_2}{a_3}{y_3}{b_1}{w_1}{b_2}{w_2}{b_3}{w_3}{b_4}{w_4}{b_5}{w_5}{c_1}{v_1}{c_2}{v_2}{c_3}{v_3}{c_4}{v_4}{c_5}{v_5}}
\systeme{  
-a_1 +5a_3 + b_1 - c_1 = 6,
 2a_2 + b_2 - c_2 = 2, 
 3a_1 + 4a_2  + b_3 - c_3= 9,  
 2a_3  + b_4 - c_4= 2,
 a_2 + b_5 - c_5= 0}\\
w_1 \geq 0, w_2 \geq 0, w_3 \geq 0, w_4 \geq 0, w_5 \geq 0,\\
v_1 \geq 0, v_2 \geq 0, v_3 \geq 0, v_4 \geq 0, v_5 \geq 0
\end{array}
$
Чтобы получить оптимальный базисный план $\lambda^o = (u^o, w^o, v^o)^T$ двойственной задачи, вспомним 
вектор потенциалов и оценки из последней итерации симплекс-метода:

$$
u^o = 
\begin{bmatrix}
	3\\0\\1\\
\end{bmatrix}
$$
$$
\Delta_1^o = 4, \Delta_2^o = 2
$$

Из этого получаем, что:
$$
w^o = \begin{bmatrix}
	4\\2\\0\\0\\0
\end{bmatrix},\quad
v^o = \begin{bmatrix}
	0\\0\\0\\0\\0
\end{bmatrix}
$$
\textbf{Ответ: } 
$\lambda^o = (u^o, w^o, v^o)^T =  
(3, 0 , 1, 4, 2, 0, 0, 0, 0, 0, 0, 0, 0)^T$
\end{document}
