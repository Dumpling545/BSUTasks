\documentclass[11pt]{article} 
\usepackage{tikz}
\usepackage{pgfplots}
\usepackage{fancyhdr}
\usepackage{sectsty}
\usepackage{amsmath}
\usepackage{amsthm}
\usepackage{amssymb}
\usepackage{mathtext}
\usepackage[T1,T2A]{fontenc}
\usepackage[utf8]{inputenc}
\usepackage[english,bulgarian,ukrainian,russian]{babel}
\usepackage{cases}
\usepackage{systeme}
\usepackage{xcolor}
\usepgfplotslibrary{fillbetween}
\usetikzlibrary{patterns}
\makeatletter
\def\mathcolor#1#{\@mathcolor{#1}}
\def\@mathcolor#1#2#3{%
  \protect\leavevmode
  \begingroup
    \color#1{#2}#3%
  \endgroup
}
\makeatother
\newenvironment{amatrix}[1]{%
  \left(\begin{array}{@{}*{#1}{c}|c@{}}
}{%
  \end{array}\right)
}
\pgfplotsset{height=14cm,compat=1.9}
\begin{document}
\textbf{Задание 2}. Решить графически следующую задачу ЛП как на максимум, так и на минимум:

Целевая функция:\quad  \textbf{13.}\  $\phi = 5x_1 + x_2 - 4x_3 - 2x_4$

Ограничения:\quad \textbf{14(6в).}\  $
\begin{array}{l}
\systeme*{  
2x_1 + x_2 - x_3  -x_4 = 6, 
x_1 - x_2 + 2x_3 + 9x_4 = 10, 
3x_1 + x_2 - 2x_3  - 9x_4 \leq 6,
18x_1 + 3x_2 +x_3  - x_4 \geq 30}\\
x_1 \geq 0,\ x_2 \geq 0, \ x_3 \geq 0, x_4 \geq 0
\end{array}
$


 \hfill

\textbf{Решение: } Используя ограничения задачи,
выразим $x_3, x_4$ через $x_1, x_2$:

\begin{equation*}
\systeme*{  
2x_1 + x_2 - x_3  -x_4 = 6, 
x_1 - x_2 + 2x_3 + 9x_4 = 10} \implies \begin{amatrix}{4}
   2 & 1 & -1 & -1 & 6\\  1 & -1 & 2 & 9 & 10
 \end{amatrix}  \implies
\end{equation*}
\begin{equation*}
\implies
\begin{amatrix}{4}
   -2 & -1 & 1 & 1 & -6\\  5 & 1 & 0 & 7 & 22
 \end{amatrix}  
\implies
\begin{amatrix}{4}
   -2 & -1 & 1 & 1 & -6\\  \frac57 & \frac17 & 0 & 1 & \frac{22}7
 \end{amatrix}  \implies
\end{equation*}
\begin{equation*}
\implies
\begin{amatrix}{4}
   -\frac{19}7 & -\frac87 & 1 & 0 & -\frac{64}7\\  \frac57 & \frac17 & 0 & 1 & \frac{22}7
 \end{amatrix}  
 \implies
 \begin{cases}
x_3 =  \frac{19}7x_1  + \frac87x_2  -\frac{64}7  \geq 0,\\ 
x_4 = -\frac57x_1 - \frac17x_2 +  \frac{22}7 \geq 0
\end{cases}.
\end{equation*}

Подставим в оставшиеся ограничения полученные выражения:
\begin{equation*}
\systeme*{  
3x_1 + x_2 - 2x_3  - 9x_4 \leq 6,
18x_1 + 3x_2 +x_3  - x_4 \geq 30}\\ \implies
\begin{cases} 
3x_1 + x_2 - 2\left(\frac{19}7x_1  + \frac87x_2  -\frac{64}7\right)  - 9\left(-\frac57x_1 - \frac17x_2 +  \frac{22}7\right) \leq 6\\
18x_1 + 3x_2 +\left(\frac{19}7x_1  + \frac87x_2  -\frac{64}7\right) - \left(-\frac57x_1 - \frac17x_2 +  \frac{22}7\right) \geq 30
\end{cases}
\end{equation*}
\begin{equation*}
 \implies
\begin{cases} 
x_1  \leq 4\\
75x_1 + 15x_2  \geq 148
\end{cases}
\end{equation*}
Также подставим выражения для $x_3, x_4$ в целевую функцию: 

\begin{equation*}
\phi = 5x_1 + x_2 - 4x_3 - 2x_4 = \frac17(-31x_1 - 23x_2 + 212)
\end{equation*}

Таким образом наша задача ЛП сводится к задаче ЛП меньшей размерности:

Целевая функция:\quad  $\phi_0 = -31x_1 - 23x_2$

\hfill

Ограничения:\quad $
\begin{array}{l}
\systeme*{  
\mathcolor{red}{75x_1 + 15x_2  \geq 148}, 
\mathcolor{green}{19x_1  +8x_2  \geq 64}, 
\mathcolor{blue}{5x_1 + x_2 \leq 22}}\\
0\leq \mathcolor{brown}{x_1 \leq 4},\ 0 \leq x_2 
\end{array}$

\hfill

Изобразив на графике множество планов и 
вектор нормали целевой функций, мы видим, что решением
задачи ЛП на минимум будет точка пересечения оси $Ox_2$ и
прямой $ \mathcolor{blue}{5x_1 + x_2 = 22}$, а решением задачи
ЛП на максимум -- точка пересечения оси $Ox_1$ и прямой \  $\mathcolor{green}{19x_1  +8x_2  = 64}$.

\hfill

\begin{tikzpicture}
\begin{axis}[
        xmin=-1, xmax=25,
        ymin=-1, ymax=25,
        axis lines = center,
        axis equal,
        xlabel = $x_1$,
        ylabel = {$x_2$},
      ]  
      \addplot[red, samples=200,  domain=-1:25,  name path=A] {(148 - 75*x)/15}; 
      \addplot[green, samples=200,  domain=-1:25, name path=B] {(64 - 19*x)/8}; 
      \addplot[blue, samples=200, domain=-1:25, name path=C] {22 - 5*x}; 
      \addplot[black, samples=200,  domain=-1:25,  name path=xAxis] {0}; 
      \addplot +[brown, samples=200, mark=none] coordinates {(4, -1) (4, 25)};
     \addplot[gray, pattern=north west lines] fill between[of=C and xAxis, soft clip={domain=3.368421052:4}];
    \addplot[gray, pattern=north west lines] fill between[of=C and B, soft clip={domain=0.71111111:3.368421052}];
     \addplot[gray, pattern=north west lines] fill between[of=A and C, soft clip={domain=0:0.71111111}];
     \addplot[violet, samples=100, domain=-1:25, name path=E] {23/31*x}; 
     \addplot[violet,<-] coordinates
           {(3,23/31*3) (4,23/31*4)};
      \addplot[violet,<-] coordinates
           {(7,23/31*7) (8,23/31*8)};
      \addplot[violet,<-] coordinates
           {(11,23/31*11) (12,23/31*12)};
     \addplot[violet,<-] coordinates
           {(15,23/31*15) (16,23/31*16)};
         \addplot[violet,<-] coordinates
           {(19,23/31*19) (20,23/31*20)};
     \addplot[violet,<-] coordinates
           {(23,23/31*23) (24,23/31*24)};
     \addplot[violet, dashed,  samples=100, domain=-1:25, name path=F] {-31/23*x + 22}; 
     \addplot[violet, dashed,  samples=100, domain=-1:25, name path=G] {-31/23*x + 4.54004576659}; 
     \node[label={0:\textcolor{violet}{min}}, circle,fill,inner sep=1.5pt, violet] at (axis cs:0, 22) {};
     \node[label={135:\textcolor{violet}{max}}, circle,fill,inner sep=1.5pt, violet] at (axis cs:64/19, 0) {};
    \end{axis}
\end{tikzpicture}

Найдём точку минимума: 
\begin{equation*}
\systeme*{x_1 = 0, 5x_1 + x_2 = 22} \implies \begin{cases}
x_1 = 0,\\ x_2 = 22\end{cases}
\end{equation*}

и максимума:
\begin{equation*}
\systeme*{x_2 = 0, 19x_1  +8x_2  = 64} \implies \begin{cases}
x_1 = \frac{64}{19},\\ x_2 = 0\end{cases}
\end{equation*}

\hfill

Теперь вычислим $x_3, x_4$ в точке минимума, используя ранее полученные выражения:

\begin{equation*}
\begin{cases}
x_3 =  \frac{19}7\cdot{}0 + \frac87\cdot{}22  -\frac{64}7  =  16,\\ 
x_4 = -\frac57\cdot{}0 - \frac17\cdot{}22 +  \frac{22}7 = 0
\end{cases}.
\end{equation*}
Аналогично для максимума:

\begin{equation*}
\begin{cases}
x_3 =  \frac{19}7\cdot{}\frac{64}{19} + \frac87\cdot{}0  - \frac{64}7  =  0,\\ 
x_4 = -\frac57\cdot{}\frac{64}{19} - \frac17\cdot{}0 +  \frac{22}7 = \frac{14}{19}
\end{cases}.
\end{equation*}

\hfill

\textbf{Ответ: } $min = (0; 22; 16; 0),\  max = (\frac{64}{19}; 0; 0;  \frac{14}{19})$
\end{document}
