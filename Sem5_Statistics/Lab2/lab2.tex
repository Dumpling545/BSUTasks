\documentclass[12pt]{article}

\usepackage{graphicx}
\usepackage{listings}
\usepackage{amsmath}
\usepackage{amssymb}
\usepackage[T2A]{fontenc}
\usepackage[utf8]{inputenc}
\usepackage[russian]{babel}
\usepackage[margin=1.5cm,portrait,a4paper]{geometry}
\usepackage[thinc]{esdiff}
\begin{document}

\begin{titlepage}
    \begin{center}
        \vspace*{8cm}
 
        \textbf{Отчет о проделанной работе (Вариант 2)}
 
        \vspace{0.5cm}
            Теория вероятностей и математическая статистика
             
        \vspace{1.5cm}
 
        \textbf{Михаил Басанец, Ян Кордияко}
 
        \vfill
             
             
        \vspace{0.8cm}
      
        Белорусский Государственный Университет \\
        Факультет Прикладной Математики и Информатики\\
        2020
             
    \end{center}
\end{titlepage}

\section{Условия}

Рассматриваются распределение Пуассона $\Pi(\lambda)$ 
и выборки из него.

$n = 60, \lambda = 4$ 

\section{Теория: }

Начальный момент распределения 1-го порядка:
$$ \alpha_1(\lambda) = \lambda $$
Начальный момент распределения 2-го порядка:
$$ \alpha_2(\lambda) = \lambda^2 + \lambda $$

\subsection{Метод моментов}

Так как функция распределения Пуассона зависит лишь от одного параметра $\lambda$,
то для получения оценки этого параметра по методу моментов  $\hat{\lambda} = T(X)$  достаточно решить уравнение:

$$ \alpha_1(\lambda) = a_1,\quad a_1 = \frac1n \sum_{i=1}^n{x_i} $$

Таким образом, оценка примет вид $\hat{\lambda} = \frac1n \sum_{i=1}^n{x_i}$.

\subsection{Свойства оценки параметра по методу моментов}


Учитывая, что $\alpha_1(\lambda) = \lambda$ -- взаимо-однозначная непрерывная функция,  
найденная оценка $\hat{\lambda}$ будет \textbf{состоятельной}.
В силу того, что:
\begin{alignat*}{3}
      \diff*{\alpha_1}{\lambda}{\lambda = 4}  = 1 \ne 0
 \end{alignat*}
найденная оценка  $\hat{\lambda}$ будет \textbf{асимптотически нормальной}.
\section{Результаты работы программы}

начальный момент 1 порядка:  4 \\
начальный момент 2 порядка:  20


\subsection{Выборка размером n:}
значение оценки ММ $\hat{\lambda}$:  3.6166666666666667 \\

\subsection{Выборка размером 3n:}
значение оценки ММ $\hat{\lambda}$:  4.116666666666666 \\

\subsection{Выборка размером 6n:}
значение оценки ММ $\hat{\lambda}$:  3.977777777777778 \\

\subsection{Выборка размером 12n:}
значение оценки ММ $\hat{\lambda}$:  4.036111111111111 \\

\section{Выводы}
Можно заметить, что при увеличении размера выборки значение оценки ММ приближается к точному
значению параметра $\lambda$, поэтому
для более точных исследований необходимо использовать выборки большего объема.
\section{Исходный код программы}

\begin{lstlisting}
import numpy as np
n = 60
l = 4
print( l)
print( l*l + l)
for i in [1, 3, 6, 12]:  
    sample = np.random.poisson(lam=l, size=i*n)
    print("size = ",i,"n: ", sum(sample) / len(sample))
\end{lstlisting}

\end{document}